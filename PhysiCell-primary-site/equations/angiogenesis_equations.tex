\documentclass[10point]{article}

\usepackage{epsfig}
% \usepackage{eps2pdf}
\usepackage{graphicx}

\usepackage{amsmath}
\usepackage{amssymb}

\usepackage[letterpaper,margin=0.5in]{geometry}

\newcommand{\beqa}{\begin{eqnarray}}
\newcommand{\eeqa}{\end{eqnarray}}
\newcommand{\beq}{\begin{equation}}
\newcommand{\eeq}{\end{equation}}

\newcommand{\abs}[1]{\left|#1\right|}
\newcommand{\norm}[1]{\left|\left|#1\right|\right|}
\newcommand{\ip}[2]{\left\langle#1,#2\right\rangle}

\newcommand{\oxy}{\textrm{pO}_2}
\newcommand{\oxyS}[1]{\textrm{pO2}_{2,\textrm{#1}}}

\renewcommand{\vec}[1]{\mathbf{#1}}
\renewcommand{\max}[2]{\textrm{max}\left(#1,#2\right)}
\renewcommand{\min}[2]{\textrm{min}\left(#1,#2\right)}

\setlength{\parindent}{0pt}


\begin{document}

\begin{center}\textbf{Angiogenesis Equations} \end{center}

Start with general mass conservation with a flux and source/sink term.

\beq
     \frac{\partial \rho_{\textrm{v}}}{\partial t} = -\vec{\nabla} \cdot \vec{J} + \textrm{S}
\eeq

In our case - 
% should letters in equations and math blocks be italized???????????
\beq
 \vec{J} = \mu_{\textrm{v}} \rho_{\textrm{v}} \frac{\vec{\nabla a}}{\parallel \vec{\nabla a} \parallel} \textrm{v}_{\textrm{m}} 
\eeq

where $\mu_{\textrm{v}}$ is the maximum vascular migration rate, $\rho_{\textrm{v}}$ is the vascular density, $\vec{\nabla a}$ is the gradient of the angiogenic factor $\textrm{a}$, and $\textrm{v}_{\textrm{m}}$ the vascular migration rate as a function of the local concentration of the angiogenic factor.  
\newline

And for the source and sink term:


\beq
\textrm{S} = \overbrace{\beta_{\textrm{v}} (1 - \frac{\rho_{\textrm{v}}}{\overline{\rho}_{\textrm{v}}}) \rho_{\textrm{v}}\textrm{b}_{\textrm{v}}}^{ \textrm{Vascular Birth}} - \overbrace{\textrm{d}_{\textrm{v}} \frac{\rho_{\textrm{t}}}{\overline{\rho}_{\textrm{t}}} \rho_{\textrm{v}}}^{ \textrm {Vas. Death}}
\eeq

where $\beta$ is the maximum vascular birth rate,  $\rho_{\textrm{v}}$ is the vascular density growing with logistic growth,  a birth rate $\textrm{b}_{\textrm{v}}$ determined by concentration of the angiogenic factor (see below) and a vascular death term determined by a constant death rate $\textrm{d}_{\textrm{v}}$, the presence of tumor tissue, and the presence of vasculature. 
\newline

Combining them all together ... 


\beq
     \frac{\partial \rho_{\textrm{v}}}{\partial {\textrm{t}}} = \overbrace{ - \vec{\nabla} \cdot \left( \mu_{\textrm{v}} \rho_{\textrm{v}} \frac{\vec{\nabla a}}{\parallel \vec{\nabla a} \parallel} \textrm{v}_{\textrm{m}} \right)}^{ \textrm{Advection} } + \overbrace{\beta (1 - \frac{\rho_{\textrm{v}}}{\overline{\rho}_{\textrm{v}}}) \rho_{\textrm{v}} \textrm{b}_\textrm{v}}^{ \textrm{Vascular Birth}} -\overbrace{\textrm{d}_{\textrm{v}} \frac{\rho_{\textrm{t}}}{\overline{\rho}_{\textrm{t}}} \rho_{\textrm{v}}}^{ \textrm {Vas. Death}}
\eeq
\newline
With the following "sub-functions" for environmentally controlled vascular chemotaxis and growth.
\newline
\newline
Chemotaxis modification:
\beq
v_\textrm{m} = 
\left\{
\begin{array}{lr} 
1 & \textrm{ if } a > a_\textrm{{saturation,chem}} \\ 
\\
\frac{ a - a_\textrm{{threshold,chem}}}{a_\textrm{{saturation,chem}} - a_\textrm{{threshold,chem}}} & \textrm{ if } a_\textrm{{threshold,chem}} \le a \le a_\textrm{{saturation,chem}} \\
\\
0 & \textrm{ if } a < a_\textrm{{threshold,chem}}
\end{array}
\right.
% \left( \frac{ \sigma_H- \sigma  }{\sigma_H - \sigma_C }  \right)^+
\eeq

Growth rate modification:

\beq
b_\textrm{{rate}} = 
\left\{
\begin{array}{lr}
1 & \textrm{ if } a > a_\textrm{{saturation,prol}} \\ 
\\
\frac{ a - a_\textrm{{threshold,prol}}}{a_\textrm{{saturation,prol}} - a_\textrm{{threshold,prol}}} & \textrm{ if } a_\textrm{{threshold,prol}} \le a \le a_\textrm{{saturation,prol}} \\
\\
0 & \textrm{ if } a < a_\textrm{{threshold,prol}}
\end{array}
\right.
% \left( \frac{ \sigma_H- \sigma  }{\sigma_H - \sigma_C }  \right)^+
\eeq
\newline
Where the there are various angiogenic concentration parameters that can be set to make a "ramp" or sigmoidal function to modify the rates movement and growth rates between 0 (a threshold) and 1 (saturation/max).  
\newline

Finally, in the prototype model, we used a threshold to remove vascular density completely from a voxel.  This is a condition such that if the vascular density goes below a threshold, the value at that point is set to 0.  This is due to the numerics allowing a point in space to carry a non-biological amount of vasculature, which under the right conditions, could regrow instead of having new vasculature chemotax there.

	
\end{document}
