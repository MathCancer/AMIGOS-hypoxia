\documentclass[12pt]{article}

\usepackage{amsmath}
\usepackage{amssymb}
\usepackage[letterpaper,margin=0.5in,bottom=0.75in]{geometry}

\renewcommand{\vec}[1]{\mathbf{#1}}
\newcommand{\oxy}{  {\textrm{pO}_2} }
\newcommand{\oxyS}[1]{ \textrm{pO}_{2\textrm{,#1}} }
\newcommand{\mmHg}{\: \textrm{mmHg} }
\newcommand{\micron}{\:\mu\textrm{m} }

\usepackage{graphicx}

\pagestyle{plain}

\begin{document}

\section{Assumptions and Questions}
\begin{enumerate}

\item 
A: All cells initially express RFP

\item 
A: Oxygen below 10 mmHg turns off expression of RFP, turns on expression of GFP

\item 
Q: What's a time scale for protein synthesis? (Early on, it's time to transcribe RNA, then synthesize protein. Later, it's time to just synthesize protein from already existent RNA.) 

\item 
Q: What's the time scale for protein degradation? 

\item 
Q: In 10 mmHg, does RFP gene get snipped out immediately and GFP gene enabled, or is there a mean time delay? 


\end{enumerate}

\section{Model}

\subsection{Gene - Protein network}
We will model a set of genes $\vec{G}$ that encode proteins $\vec{P}$ with the following model: 
\begin{eqnarray}
\frac{ d P_i }{dt} & = & \alpha_i G_i - \beta_i P_i, \hspace{.25in} i = 1, 2, \ldots 
\end{eqnarray}
where $\alpha_i$ is a protein creation rate, and $\beta_i$ is a protein degradation rate. (Notice that this 
skips modeling RNA transcription.) Here, we will model the following genes: \\

\begin{center}
\begin{tabular}{c|c|l}
index & protein & notes \\
 \hline 
 0 & RFP & default fluorescence \\
 1 & GFP & activated at $\oxy = 10 \mmHg $ 
\end{tabular}
\end{center}

Gene expression can be modeled in any way. Here, we set $G_1 =1$ if $\oxy < 10 \mmHg$. 

\subsubsection{Parameter estimates}



\end{document}


