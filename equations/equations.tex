\documentclass[12pt]{article}

\usepackage{amsmath}
\usepackage{amssymb}
\usepackage[letterpaper,margin=0.5in,bottom=0.75in]{geometry}

\renewcommand{\vec}[1]{\mathbf{#1}}
\newcommand{\oxy}{  {\textrm{pO}_2} }
\newcommand{\oxyS}[1]{ \textrm{pO}_{2\textrm{,#1}} }
\newcommand{\mmHg}{\: \textrm{mmHg} }
\newcommand{\micron}{\:\mu\textrm{m} }

\usepackage{graphicx}

\newcommand{\beqa}{\begin{eqnarray}}
\newcommand{\eeqa}{\end{eqnarray}}

\pagestyle{plain}

\begin{document}

\section{Assumptions and Questions}
\begin{enumerate}

\item 
A: All cells initially express RFP

\item 
A: Oxygen below 10 mmHg turns off expression of RFP, turns on expression of GFP

\item 
Q: What's a time scale for protein synthesis? (Early on, it's time to transcribe RNA, then synthesize protein. Later, it's time to just synthesize protein from already existent RNA.) 

\item 
Q: What's the time scale for protein degradation? 

\item 
Q: In 10 mmHg, does RFP gene get snipped out immediately and GFP gene enabled, or is there a mean time delay? 


\end{enumerate}

\section{Model}

\subsection{Gene - Protein network}
We will model a set of genes $\vec{G}$ that encode proteins $\vec{P}$ with the following model: 
\begin{eqnarray}
\frac{ d P_i }{dt} & = & \alpha_i G_i - \beta_i P_i, \hspace{.25in} i = 1, 2, \ldots 
\end{eqnarray}
where $\alpha_i$ is a protein creation rate, and $\beta_i$ is a protein degradation rate. (Notice that this 
skips modeling RNA transcription.) Here, we will model the following genes: \\

\begin{center}
\begin{tabular}{c|c|l}
index & protein & notes \\
 \hline 
 0 & RFP & default fluorescence \\
 1 & GFP & activated at $\oxy = 10 \mmHg $ 
\end{tabular}
\end{center}

Gene expression can be modeled in any way. Here, we set $G_1 =1$ if $\oxy < 10 \mmHg$. 

\subsubsection{Nondimensionalization}
Let $\overline{P}$ be the maximum protein level with $G = 1$. Then by equilibrium analysis, 
$\overline{P} = \frac{\alpha}{\beta}$. If we nondimensionalize the main ODE form, we get 
\beqa
\frac{dP_i}{dt} & = & \beta_i \left( G_i - P_i \right). 
\eeqa
This functional form sucks, because it doesn't let us set the rate of reaching near $P_i \sim 1$ independently 
of the decay rate. Blech!

\subsubsection{Better model and nondimensionalization}
Now, suppose that $P^*$ is a protein value where negative feedback reduces either transcription or synthesis. Then 
\beqa
\frac{ dP_i }{dt } & = & \alpha_i G_i \left( P_i^* - P_i \right) - \beta_i P_i. 
\eeqa
By equilibrium analysis, the maximum protein value is 
\beqa
\overline{P}_i & =& 
\frac{\alpha_i P_i^*}{\alpha_i + \beta_i }
\eeqa
If we nondimensionalize by this, we get: 
\beqa
\frac{ dP_i }{dt} & = & 
G_i \left( \alpha_i + \beta_i \right) - \left( \alpha_i G_i + \beta_i \right) P_i \nonumber \\
& = & 
G_i \alpha_i \left( 1 - P_i \right) + \beta_i \left( G_i - P_i \right) . 
\eeqa
Notice that in this form, $\alpha_i$ sets the rate of approaching the maximum protein expression (1) 
when the gene is expressed, and $\beta_i$ sets the rate of decay. 

\subsubsection{Parameter estimates}
Let's suppose for now that it takes about 10 minutes to ramp up a protein level to 90\% of its 
maximum value. Then in the absence of degradation, we have 
\beqa
\frac{dP_i}{dt} & \approx & G_i \alpha_i \left( 1 - P_i \right). 
\eeqa
So, if we want to reach $P_i = 0.9$ after $T_{S,i} = 10 \textrm{ min}$ time, then 
\beqa
\alpha_i &  =&  -\frac{ \ln{10} }{ T_{S,i} } \approx 0.23  \textrm{ min}^{-1}. 
\eeqa
Similarly, if $G_i = 0$, we can easily fit $\beta_i$ by a decay time scale $T_{D,i}$, defined 
for us to be the time to degrade 90\% of the protein. Let's suppose for now $T_{D,i} = 120 \textrm{ min}$. 
Then 
\beqa
\frac{dP_i}{dt} & \approx & - \beta_i  P_i  
\eeqa
and so 
\beqa
\beta_i & = & -\frac{ \ln{10} }{ T_{D,i} } \approx 0.019 ? \textrm{ min}^{-1}. 
\eeqa

\subsubsection{Simple implicit numerical scheme}
Suppose that $P_i^n = P_i( t_n ) = P_i ( t_0 + n \Delta t)$. Then 
\beqa
P^{n+1}_i 
& = & 
\frac{ P_i^n + \Delta t G_i^n \left( \alpha_i + \beta_i \right) }{ 1 + \Delta t \left( \alpha_i G_i^n + \beta_i \right)}
\eeqa



\end{document}


